\documentclass{article}
\usepackage{epsfig}
\usepackage{graphicx}
\usepackage[top=0.50in, bottom=0.50in, left=0.65in, right=0.75in]{geometry}
%\usepackage[a4paper, total={6in, 10in}]{geometry}
\usepackage[table]{xcolor}
\usepackage{tikz}
\usepackage{algorithm}
\usepackage{mathtools}
\usepackage{amsmath,amssymb}
\usepackage[]{algpseudocode}
\usepackage{enumitem}
\title{CS345 Theoretical Assignment 5 \\ }
\author{\vspace{2mm} \large Ayush Agarwal, 13180 \\ M.Arunothia, 13378}
\date{}
\begin{document}
\maketitle
\tableofcontents

\newpage
\section{Binary search under insertion}
\subsection{Data Structure Overview}
The data structure consists of arrays of multiple arrays with size of the form $2^i$. For searching, we search each of these arrays. To insert an element,
we first insert it into the first array(of size 1). If it is full we try to insert them into array of size 2. If that is also full we try the same for array 
of size 4 and so on. This insertion operation is amortized $logn$.
\begin{itemize}
  \item \textbf{array}[i] is the pointer to array of size $2^i$
  \item \textbf{filled} number of arrays being used
\end{itemize}

\begin{algorithmic}[1]
\Procedure{\textbf{insertion}(int newElement)}{}
    \State i $\gets$ 0
    \While{!array[i].empty()}
    \State $i$++
    \EndWhile
    \State array[i] $\gets$ \textbf{merge}(array, i, newElement)
    \If{i $>=$ filled}
      \State filled $\gets$ i
    \EndIf
  \EndProcedure
  \\
  \Procedure{\textbf{merge}(int *array[], int end, int newElement)}{}
    \State Merge all the arrays with lengths starting from 1 to $2^end$ and insert the new element into that array.\\
    \State i $\gets$ 0
    \While{i $<$ end}
      \State free(array[i])
      \State $i$++
    \EndWhile
    \State return mergedArray
  \EndProcedure
  \\
  \Procedure{\textbf{search}(int element)}{}
    \State i $\gets$ 0
    \While{i $<=$ filled}
    \If {binarySearch(array[i], element)}
    \State return True
    \EndIf
    \State i $\gets$ i + 1
    \EndWhile
  \EndProcedure

\end{algorithmic}

\subsection{Justification}
At any point of time, any of the $array[i]$ will either be empty or completely filled. There can't be a case when any of the array is partially filled. Think of 
it like the binary number representation of number n. The indexes where n has value 1, those corresponding arrays are filled, rest are empty.

\subsection{Time Complexity Analysis}
\begin{itemize}
  \item \textbf{search} Worst case time would be when every array from size 1 to size $2^{logn-1}$ would be filled. Without the loss of generality assume n = $2^k$
    So search time,\\
    = log1 + log2 + $.....$ logn/2 \\
    = (1 + 2 + 3 $...$ k-1) \\
    = k(k-1)/2 \\
    = O($log^2$n) \\
    \\
  \item \textbf{insert}\\
    Let amortized function $$\phi(i) = \sum_{e}^{all elements}depth(e)  = \sum_{i}^{i \leq logn} 2^i(logn - i) $$ (where i is all the nonzero indices in the binary representation of n) after the $i^th$ insertion where \\
    $depth(e) = logn - arrayIndex(e)$.\\
    \begin{tabular}{|p{4cm}|p{3cm}|p{3cm}|p{3cm}|  }
      \hline
      \multicolumn{4}{|c|}{Amortised Analysis} \\
      \hline
      Case & Actual Cost& $\Delta(\phi)$ & Amortised Cost\\
      \hline
      Direct Insert(no merge) & 1           & logn & O(logn)   \\
      On merge(till array $2^k$) & $2^{k+1}$   & $-\sum_{i}^{i<k} (2^i(k-i)) = -(2^{k+1} -k - 2)$ & k+2 $\leq$ O(logn) \\
      \hline
    \end{tabular}

\end{itemize}

\newpage
\section{Binary search and predecessor/successor queries under deletions}
\subsection{Data Structure Overview}

\newpage    
\section{Extension of the problem of the mid-semester exam}
\subsection{Pseudo Code}
\begin{algorithmic}[1]
  \Procedure{\textbf{findSubgraph}(V,E)}{}
  \State $E_s \gets \phi$
  \While{$V != \phi$}
  \State Pick any vertex $v$ from $G$ 
  \If {$degree(v) <= n^{1/k}$}
  \State Add all edges incident on $v$ to $E_s$
  \State Remove $v$ and all edges incident on $v$ from $G$
  \Else
  \State Do a BFS from $v$ in graph $G$ till a depth of $k-1$.  
  \State Remove all the non-tree edges (in the formed $k-1$ levels) from $G$
  \State Call this formed tree of depth $k-1$ as $cluster$. 
  \State For all $v \not\in cluster$ retain just one edge with the cluster and remove all other edges from $G$
  \State Add all the tree(cluster) edges along with the edges incident on the cluster to $E_s$
  \State Remove $cluster$ and all edges incident on $cluster$ from $G$
  \EndIf
  \EndWhile
  \State return $Es$
  \EndProcedure
\end{algorithmic} 
\subsection{Justifications}
\subsubsection{Retaining just one edge with the cluster for any $v \not\in cluster$ is sufficient}
Let $v$ be the root of the $cluster$ being discussed. Let $u \not \in cluster$ be the vertex outside cluster who has edges to both $x \in cluster $ and $y \in cluster$. Let us consider what happens when we remove say the edge $(u,y)$. We know $v$ being the root of the BFS tree, is connected to both $x$ and $y$. As the depth of the BFS tree being considered is $k-1$, the maximum path length between $v$ and $x$(or $y$) is $k-1$. This means there is a path between $x$ and $y$ via $v$ that has a maximum length of $2(k-1)$. Though we have removed the edge $(u,y)$, $u$ and $y$ are still connected by the path $(u,x)$, then $x$ to $y$ via $v$. The maximum length of this path between $u$ and $y$ is hence, $2k-1$, satisfying the requirement asked for in the question. This explains why retaining just one edge with the cluster for any $v \not\in cluster$ is sufficient.
\subsubsection{Non-Tree edges need not be retained}
As within the tree any two vertices are always connected via a path whose length $\leq 2k-1$, there is no need to retain non-tree edges.

\subsection{Proving $|E_s| = O(n^{1+1/k}) $}
\begin{itemize}
\item If $degree(v) <= n^{1/k}$ then, we add atmost $n^{1/k}$ edges to $E_s$ for the single vertex $v$.
\item If $degree(v) > n^{1/k}$ then, we add atmost $O($ $vertices$ $being$ $removed$ $ + $ $vertices$ $not$ $being$ $removed$ $)$ edges to $E_s$ for a $cluster$ of atleast $n^{1/k}$ vertices.
\begin{itemize}
\item The number of edges that belong to the $cluster$ is of $O($ $vertices$ $being$ $removed$ $)$ as it is a tree.
\item The number of edges that are incident on the cluster is $O($ $vertices$ $not$ $being$ $removed$ $)$ as proved from justification (1). 
\item Hence, the given order.
\end{itemize}
\item Worst case $|E_s|  =  n*n^{1/k} + \sum_{i=0}^{k} n^{1-i/k}$ = $O(n^{1+1/k})$ 
\end{itemize}
\subsection{Time Complexity}
The overall algorithm accesses an edge exactly for O(1) times, because in any iteration of the while-loop the edge getting accessed is being removed off from $G$ and hence, it is guaranteed that no edge is being accessed in two different iterations. Hence, the time complexity is $O(m+n)$.    
\end{document}
