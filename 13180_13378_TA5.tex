\documentclass{article}
\usepackage{epsfig}
\usepackage{graphicx}
\usepackage[top=0.50in, bottom=0.50in, left=0.65in, right=0.75in]{geometry}
%\usepackage[a4paper, total={6in, 10in}]{geometry}
\usepackage[table]{xcolor}
\usepackage{tikz}
\usepackage{algorithm}
\usepackage{mathtools}
\usepackage{amsmath,amssymb}
\usepackage[]{algpseudocode}
\usepackage{enumitem}
\title{CS345 Theoretical Assignment 5 \\ }
\author{\vspace{2mm} \large Ayush Agarwal, 13180 \\ M.Arunothia, 13378}
\date{}
\begin{document}
\maketitle
\tableofcontents
\newpage
\section{Binary search and predecessor/successor queries under deletions}
\subsection{Data Structure Overview}

\newpage    
\section{Extension of the problem of the mid-semester exam}
\subsection{Pseudo Code}
\begin{algorithmic}[1]
  \Procedure{\textbf{findSubgraph}(V,E)}{}
  \State $E_s \gets \phi$
  \While{$V != \phi$}
  \State Pick any vertex $v$ from $G$ 
  \If {$degree(v) <= n^{1/k}$}
  \State Add all edges incident on $v$ to $E_s$
  \State Remove $v$ and all edges incident on $v$ from $G$
  \Else
  \State Do a BFS from $v$ in graph $G$ till a depth of $k-1$.  
  \State Remove all the non-tree edges (in the formed $k-1$ levels) from $G$
  \State Call this formed tree of depth $k-1$ as $cluster$. 
  \State For all $v \not\in cluster$ retain just one edge with the cluster and remove all other edges from $G$
  \State Add all the tree(cluster) edges along with the edges incident on the cluster to $E_s$
  \State Remove $cluster$ and all edges incident on $cluster$ from $G$
  \EndIf
  \EndWhile
  \State return $Es$
  \EndProcedure
\end{algorithmic} 
\subsection{Justifications}
\subsubsection{Retaining just one edge with the cluster for any $v \not\in cluster$ is sufficient}
Let $v$ be the root of the $cluster$ being discussed. Let $u \not \in cluster$ be the vertex outside cluster who has edges to both $x \in cluster $ and $y \in cluster$. Let us consider what happens when we remove say the edge $(u,y)$. We know $v$ being the root of the BFS tree, is connected to both $x$ and $y$. As the depth of the BFS tree being considered is $k-1$, the maximum path length between $v$ and $x$(or $y$) is $k-1$. This means there is a path between $x$ and $y$ via $v$ that has a maximum length of $2(k-1)$. Though we have removed the edge $(u,y)$, $u$ and $y$ are still connected by the path $(u,x)$, then $x$ to $y$ via $v$. The maximum length of this path between $u$ and $y$ is hence, $2k-1$, satisfying the requirement asked for in the question. This explains why retaining just one edge with the cluster for any $v \not\in cluster$ is sufficient.
\subsubsection{Non-Tree edges need not be retained}
As within the tree any two vertices are always connected via a path whose length $\leq 2k-1$, there is no need to retain non-tree edges.

\subsection{Proving $|E_s| = O(n^{1+1/k}) $}
\begin{itemize}
\item If $degree(v) <= n^{1/k}$ then, we add atmost $n^{1/k}$ edges to $E_s$ for the single vertex $v$.
\item If $degree(v) > n^{1/k}$ then, we add atmost $O($ $vertices$ $being$ $removed$ $ + $ $vertices$ $not$ $being$ $removed$ $)$ edges to $E_s$ for a $cluster$ of atleast $n^{1/k}$ vertices.
\begin{itemize}
\item The number of edges that belong to the $cluster$ is of $O($ $vertices$ $being$ $removed$ $)$ as it is a tree.
\item The number of edges that are incident on the cluster is $O($ $vertices$ $not$ $being$ $removed$ $)$ as proved from justification (1). 
\item Hence, the given order.
\end{itemize}
\item Worst case $|E_s|  =  n*n^{1/k} + \sum_{i=0}^{k} n^{1-i/k}$ = $O(n^{1+1/k})$ 
\end{itemize}
\subsection{Time Complexity}
The overall algorithm accesses an edge exactly for O(1) times, because in any iteration of the while-loop the edge getting accessed is being removed off from $G$ and hence, it is guaranteed that no edge is being accessed in two different iterations. Hence, the time complexity is $O(m+n)$.    
\end{document}